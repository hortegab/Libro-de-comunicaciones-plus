%%%% INTRODUCCION
\chapter*{\centering INTRODUCCIÓN}
\markright{INTRODUCCION}
\addcontentsline{toc}{chapter}{INTRODUCCIÓN}%para que aparezca en la tabla de contenidos
\label{Cap:Introduccion}


\markright{INTRODUCCIÓN}

Las comunicaciones han sido durante todos los tiempos la clave para el desarrollo social de la humanidad. Desde las señales de humo como medio de comunicación, pasando por la invención de la rueda, la imprenta, hasta llegar al teléfono, la televisión y el Internet. Cada uno de esos avances ha impulsado un nuevo paso del hombre por las diferentes etapas de desarrollo social como la sociedad agraria, la sociedad industrial y la sociedad de la información. Por eso, varios autores aseguran que la historia de las comunicaciones es la historia de las revoluciones sociales. El nuevo paradigma es hoy la convergencia de las comunicaciones, es decir, la unión de: los medios de comunicación; la robótica; las redes de sensores para medir diferentes variables y fenómenos; el procesamiento inteligente de la información para la toma de decisiones; la informática; elementos normativos, de negocios y sociales. Esto al servicio de las actividades humanas como por ejemplo: la medicina, la industria y la conservación del medio ambiente. El término que aglutina todos esos componentes es hoy “\textit{Tecnologías de la Información y las Comunicaciones} (TIC)”. La ola de fusiones estratégicas a nivel empresarial así como los acontecimientos mundiales que se relacionan con la globalización del mundo son consecuencia de las TIC.
\\

El auge de los sistemas inalámbricos y el desarrollo de más y mejores servicios de última generación en comunicaciones exige a los ingenieros electrónicos % \OR sólo a los ing electrònicos?
estar a la vanguardia de estos avances científicos y tecnológicos. Ante esta realidad, este curso busca crear las competencias necesarias para que los ingenieros electrónicos egresados de la UIS 
% \OR resulta confusa la idea. El libro se dirige EGRESADOS? ¿sólo de la UIS? ¿sólo ing. electrónicos?
puedan fácilmente comprender las telecomunicaciones no sólo al nivel que requieren los operadores y proveedores de tecnologías, sino también al nivel científico. Abordar todos los temas de las comunicaciones en este curso % \OR ¿este libro en lugar de este curso?
es prácticamente imposible, pero sí se enfatiza en las competencias clave para lograr que los ingenieros de la UIS tengan una ventaja teórico-práctica frente a otros egresados para facilitar su participación en la demanda nacional e internacional en el campo de las comunicaciones. % \OR de nuevo, no considero apropiado el énfasis en que el libro se dirija sólo a ingenieros de la UIS.
Son temas de estudio de este curso: 
% \OR de nuevo: curso o libro?
los sistemas de comunicaciones digitales, analizando diferentes sistemas de modulación digital pasobanda y bandabase actuales y profundizando en las técnicas de acceso al medio más importantes hoy en día. De igual forma se estudian diversas aplicaciones de uso actual, sus principios y fundamentos necesarios para comprenderlas y analizarlas. Se hace uso de la programación de sistemas basados en SDR usando GRC y USRP como enfoque práctico de estos temas.
% \OR el glosario debería ser anterior al uso de siglas. En todo caso, es recomendable escribir su significado, al menos la primera vez que se usan.
Aunque se abordan muy débilmente otros temas relevantes como planeación de redes, mercados de las TIC, normativa y gestión del espectro, puede decirse que con este curso % \OR curso o libro?
el estudiante habrá desarrollado las competencias que mayor esfuerzo intelectual y mayor acompañamiento requieren, lo cual habrá pavimentado el escabroso camino teórico-práctico para poder avanzar de manera más cómoda en la construcción de las competencias que demanda el mercado.\\ % \OR este último párrafo, y en general la introducción se pueden revisar una vez el libro esté concluido para asegurar su coherencia y consistencia.

Si nos preguntaran en qué se diferencia este libro de todos los demás, la respuesta estaría en los siguientes puntos:\\ 

\begin{itemize}
	\item [$\bullet$]  Combina la enseñanza de las comunicaciones con la de GNU Radio.
	\item [$\bullet$] Ha sido escrito como parte de un proceso real de enseñanza, de modo que está probado en varias generaciones de estudiantes, ha crecido con ellas.
	\item [$\bullet$] Está orientado al desarrollo de competencias mediante prácticas de laboratorio.
	\item [$\bullet$] Se orienta hacia una enseñanza problematizada, de modo que para cada tema se propone un problema a resolver, se analizan sus causas y se presenta el tema como la solución a ese problema.
	\item [$\bullet$] Se usa una enseñanza siempre basada en modelos de capas.
\end{itemize}
